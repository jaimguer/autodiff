\documentclass{article}

\author{Jaime Guerrero}
\date{May 6, 2015}
\title{Autodifferentiation in Racket and Lua}

\begin{document}
\maketitle

\section{Introduction}
Differentiation of functions plays a foundational role in countless domains,
including mathematics, engineering, economics, and various social sciences.


\section{Autodifferentiation Overview}
The two primary forms of automatic differentiation, reverse mode and forward
mode, both exploit the compositional nature of the chain rule.  The forms differ
in the way they approach the multiplications within the chain rule; forward mode
goes from right to left, while reverse mode goes from left to right.  Forward
mode is the more straightforward of the two, and it has been implemented here.
This implementation does not allow for functions with mulitple unknows, but can
calculate partial derivatives with respect to one unknown.  What is special
about reverse mode? This is a numerical technique.


\section{A Haskell Implementation}
The bulk of this code is inspired from Jerzy Karczmarczuk's Haskell
implementation SOURCE.  In his paper, Karczmarczuk exploits Haskell's lazy
evaluation to create a sophisticated system.  However, the first part of the
paper works perfectly well in a strict setting.  His implementation utilizes the
most common technique for forward mode differentatino: an algebra of
\textit{dual numbers}.  A dual number is a pair of numbers, wherein the first
component is a function's value at a point, and the second component is the
function's derivative at that same point. In order to instantiate a dual number,
a higher-order function \texttt{dlift} function is given some function and its
derivative.  When provided with a point, \texttt{dlift}'s function creates
a dual and calculates its derivative.

The use of dual numbers is a concise way to describe a function's value and
derivative.  But the above description of differentiation amounts to nothing
more than a large table that maps functions to their derivatives.  In order for
more sophisticated calculations, standard arithmetic operators must be
overloaded to work upon the dual numbers that \texttt{dlift} generates.  And
since duals already carry derivatives along with them, the operators are where
the standard derivative rules of calculus can be encoded.  For example,
\texttt{*} takes two dual numbers and returns a third, where the returned value
consists of multiplied function values and an encoding of the product rule as
its second component.  It is in this overloading that we can find the
derivatives of all functions consisting of elementary operators upon algebraic
and trigonometric functions.

\section{Scheme and Lua Implementations}
Operator overloading via explicit renaming
Operator overloading via metatables

\section{An Example}

\section{Takeaways}
\subsection{Operator overloading is DSL construction}
Hudak's paper?
At the highest level, we can understand operator overloading as a way to implement a DSL.  This is particularly useful in our case, where both implementations (more or less) retain their mathematical flavor.

\subsection{Derivative Construction as Metaprogramming}

\subsection{Optimizations}


\end{document}
